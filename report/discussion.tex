\section{Discussion}

In this assignment we have been introduced to the development of processors using 
industrial grade tools. The task challenged us and showed how hardware design is
very different from programming. And the workflow was quite different from what we
where used to. The process requires more planning ahead, easy hacks would not help you.
But still we could apply what we knew about testing software to hardware testing 
through testbenches. This provided a effective way to verify our design and concentrate
on the details of the implementation. 

All the required functions for the assignment has been implemented and tested. And the
overall design is clean with a separate of conserns for the three components developed. 
New instructions can easly be added and the solution can work as a starting point for the
next assignment. 

The part of the assignment we lost the most time on was from a implemented design to a running
program on the FPGA. And this part was estimated to take only 5\% of the whole process. 
After a lot of trial and error, with a long turnaround time due to the tools, we managed to get 
a working result on the FPGA. Luckly we had time in our schedule to put in some extra hours for this.
The cause of this problem seemes to have been the clock frequency of the FPGA. It was running higher 
than our core could handle and therefore no loads or store instructions had the time to complete. 
The problem was solved by removing the whole implementation completely and incrementaly adding
it as we tested on the hardware. This has prepared us for the next assignment as we will try
to get a working system early in the process and run tests on hardware throughout the development
cycle. 

As the clock frequency option was a bit hard to find we suggest to include it in the compendium
for next years assignments. 
