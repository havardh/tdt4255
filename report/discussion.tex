\section{Discussion}

In this assignment we have been introduced to the development of processors using 
industrial grade tools. The task challenged us and showed how hardware design is
very different from programming. And the workflow was quite different from what we
were used to. The process requires more planning ahead, easy hacks will not help you.
But still we could apply what we knew about testing software to hardware testing 
through testbenches. This provided an effective way to verify our design and concentrate
on the details of the implementation. 

All the required functions for the assignment has been implemented and tested. The
overall design is clean with a separations of concerns for the three components developed. 
New instructions can easly be added and the solution can work as a starting point for the
next assignment. 

The part of the assignment we lost the most time on was from an implemented design to a running
program on the FPGA. This part was estimated to take only 5\% of the whole process. 
After a lot of trial and error, with a long turnaround time due to the tools, we managed to get 
a working result on the FPGA. Luckily we had time in our schedule to put in some extra hours for this.
The cause of this problem seems to have been the clock frequency of the FPGA. It was running higher 
than our core could handle and therefore no load or store instructions had the time to complete. 
The problem was solved by removing the whole implementation completely and incrementally adding
it as we tested on the hardware. This has prepared us for the next assignment as we will try
to get a working system early in the process and run tests on hardware throughout the development
cycle. 

As the clock frequency option was a bit hard to find we suggest including it in the compendium
for next years assignments. 

In the final hours a bug relating to the PC register managed to sneak in, resulting in our processor always skipping the
first instruction in memory. As it was not immediately obvious what caused the bug, and as the time until the deadline was
running out, the bug had to be left in. This is not a major bug however, more like a minor inconvienice, as it can easly be 
solved by offsetting our programs by one address, and the rest of the processor is still functioning as intended.
