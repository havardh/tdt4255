\section{Implementation on the FPGA}

\subsection{Creating the bit file}
The final bit file {\bf system\_final.bit} is included as a deliverable for easy testing of the project on a FPGA. But the file can also easly be create by following these steps.


\begin{enumerate}
	\item Open the ISE project located at:
		{\bf system/system.xise}
	\item Select the System file in the navigator
  \item Right click on ``Update Bitstream with Processor Data`` and select ReRun All. This process takes a few minutes. 
	\item The result is a file called {\bf system\_download.bit}
\end{enumerate}


\subsection{Uploading the code to the FPGA}
Using the AvProg application installed on the LAB, the FPGA can be configured with the {\bf system\_download.bit} or {\bf system\_final.big} file.
This is done by:

\begin{enumerate}
\item Opening the application and connecting to the Development board through the COM port.
\item Selecting the {\bf system\_download.bit} file through the {\bf browse} dialog. 
\item Pressing the {\bf Configure FPGA} button.
\item Disconnecting the COM port. 
\end{enumerate}  

\subsection{Running code on the FPGA}
After the FPGA is configured with the core the {\bf driver/host.py} program can be used to run programs.

To run the {\bf toplevel\_test.hex} program execute the following commands from the root directory of the project.

Note 1: The -p 3 switch must correspont to the COM port of the test pc. Refere to python driver/host.py -h

Note 2: Because of a late bug the first instruction in the program is not executed. To work around this a {\bf nop} instruction 
is added at the start of the program

\begin{verbatim}
python driver/host.py -p 3 -i ./bin/toplevel_test_nop.hex
python driver/host.py -p 3 -c d 0x1 0x2
python driver/host.py -p 3 -c d 0x2 0xa
python driver/host.py -p 3 -c s
-- wait some time
python driver/host.py -p 3 -c s
python driver/host.py -p 3 -r mem.out
\end{verbatim}

\subsection{Result from running the program}
As in the \ref{sec:exp_res} section the expected output of the program {\bf toplevel\_test.hex} is as per table \ref{table:exp_res}.
The following file snippet shows the result from running the program on the FPGA.

\begin{verbatim}
0x0 0x0
0x1 0x2
0x2 0xa
0x3 0x0
0x4 0x0
0x5 0xc
0x6 0xc
0x7 0xc
0x8 0x600000
0x9
0xa	0x0
\end{verbatim}
