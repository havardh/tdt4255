\section{Discussion}

In this exercise we have gotten an even better introduction to the development of 
processors and other components using a hardware description langauge and industrial
grade development tools. We were able to build on our experience and knownledge from the first
exercise \cite{multicycle}, which resulted in a much smoother development process with less problems
caused by misunderstanding or wrong usage of the development tools.

We were able to build on the framework of the previous exercise and use that as a 
starting point when solving this exercise. This saved us some startup time, but in the
end most of the code for this exercise ended up being written from scratch, as the 
architectures are vastly different.

Our implementation fulfills all functional requirements set by the exercise text.
We support all the required MIPS instructions, and we have a forwarding and hazard
detection unit that together solved all data and control hazards, either by forwarding
values and allowing computation to continue, or stalling or flushing the pipeline to 
prevent wrong instructions from being executed.

In order to futher improve the efficency of the pipeline we've implemented a branch
prediction unit that will predict actions taken by branches already in the ID stage,
and that improves itself using a state machine learning algorithm. Combining this with
unconditional jumps in the ID stage gives a fairly efficient pipeline.

Our branch prediction unit has a bug that was discovered close to the deadline. The bug
can be observed in the branch testbench (section~\ref{section:branch-tb}). If two
branches follow each other in the code, both are initally in state TAKEN. Given that none 
of them are to be followed, the frist one is correctly updated to NOT\_TAKEN, but the second one
updates to TAKEN\_STRONG. As we were running out of time we were
not able to fix this, but luckily this is a non-critical bug because the only side
effect it will have is that branch predictions may in some cases be sub-optimal.

During compilation of the system there are a couple of warnings related to unconnected
signals. These are caused by signals in the pipeline registers that are sent to the
related stage but not used. These are always used by atleast one of the additional units
and the warnings are expected. We could have removed them but this would have required
adding extra complexity to the pipeline registers to filter the signals not needed by
the following stages.


\subsection{Speedup}

It is natural to compare the two processors we've implemeted during this course
as they are in many ways equal, from a ISA point of view, but with two very different
implementation approches. The obvious metric here is speed, or in this case 
cycles spent executing the same test code. During testing we ran the for testbench 
(Section~\ref{section:for-tb}) on both our designs, with the result being that
our pipelined design uses only about 70\% of the time the first multicyle design
uses. 

This is a good speedup, but as we have a pipeline with 5 stages one would
assume that a greater speedup would be achievable. The main reason the speedup is not as
good as can be is that the test case contains several jumps, which all cause a 
pipeline stall. There are also two mispredicted branches which each causes two
stalls. So even if the full program when run is only 20 instructions to be executed
the pipelined design also executes a total of 6 no-ops due to jumps, 4 no-ops due
to mis-predicted branches and one stall due to a data dependecy, for a total of
31 cycles. (Two more are needed before the data is availble in memory). Compared
to our multicycle design which uses $16*2 + 4*3 = 44$ cycles, (4 loads and stores 
and 16 normal instructions)


Had the test been more focused on arithmetic we whould however have expected an
even greater speedup.