\section{Implementation on the FPGA}

\subsection{Creating the bit file}
The final bit file {\bf system\_final.bit} is included as a deliverable 
for easy testing of the project on a FPGA. But the file can also easily 
be created by following these steps.

\begin{enumerate}
	\item Open the ISE project located at:
		{\bf system/system.xise}
	\item Select the System file in the navigator
  \item Right click on ``Update Bitstream with Processor Data`` and select ReRun All. This process takes a few minutes. 
	\item The result is a file called {\bf system\_download.bit}
\end{enumerate}


\subsection{Uploading the code to the FPGA}
Using the AvProg application installed on the lab, the FPGA can be configured 
with the {\bf system\_download.bit} or {\bf system\_final.bit} file.
This is done by:

\begin{enumerate}
\item Opening the application and connecting to the Development board through the COM port.
\item Selecting the {\bf .bit} file through the {\bf browse} dialog. 
\item Pressing the {\bf Configure FPGA} button.
\item Disconnecting the COM port. 
\end{enumerate}  

\subsection{Running code on the FPGA}
After the FPGA is configured with the core our two test programs can be executed using
the {\bf driver/run\_test.sh} shell script we have prepared. It takes two arguments,
the com port number and the name of the test to be executed (either bin/toplevel\_tb
or bin/for\_tb). The shell script will prepare the memories, execute the processor,
read back the result and validate against a excpected data set. The output will either
be {\bf success} in case of a successfull test or {\bf failure} if the test fails.

The two test programs execute the same instruction set as the test benches with the same
name.

\begin{verbatim}
> ./driver/run_test.sh 3 bin/toplevel_tb
< ....
< success
\end{verbatim}